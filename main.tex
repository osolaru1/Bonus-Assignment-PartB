\documentclass[twoside]{article}

\usepackage[accepted]{aistats2021}
% If your paper is accepted, change the options for the package
% aistats2021 as follows:
%
%\usepackage[accepted]{aistats2021}
%
% This option will print headings for the title of your paper and
% headings for the authors names, plus a copyright note at the end of
% the first column of the first page.

% If you set papersize explicitly, activate the following three lines:
%\special{papersize = 8.5in, 11in}
%\setlength{\pdfpageheight}{11in}
%\setlength{\pdfpagewidth}{8.5in}

% If you use natbib package, activate the following three lines:
%\usepackage[round]{natbib}
%\renewcommand{\bibname}{References}
%\renewcommand{\bibsection}{\subsubsection*{\bibname}}

% If you use BibTeX in apalike style, activate the following line:
%\bibliographystyle{apalike}

\begin{document}

% If your paper is accepted and the title of your paper is very long,
% the style will print as headings an error message. Use the following
% command to supply a shorter title of your paper so that it can be
% used as headings.
%
%\runningtitle{I use this title instead because the last one was very long}

% If your paper is accepted and the number of authors is large, the
% style will print as headings an error message. Use the following
% command to supply a shorter version of the authors names so that
% they can be used as headings (for example, use only the surnames)
%
%\runningauthor{Surname 1, Surname 2, Surname 3, ...., Surname n}

\twocolumn[

\aistatstitle{An Effective Security Requirements Engineering
Framework for Cyber-Physical Systems Analysis
}

\aistatsauthor{ Omolola Solaru  }

\aistatsaddress{ Georgia State University,Fall 2020,CSC 4350} ]

\begin{abstract}
    
In this present technological era,this article details how Cyber-Physical Systems (CPSs) are placing priority over other frameworks of systems through the utilization of a Security Goals of Cyber-Physical System. The diverseness of these systems builds the significance of security. Seeing how both the developer and the requirement analyst must think about the design of the product, in addition the hardware 
viewpoint, which includes sensor and network security. As a result a few models were developed for the protected software engineering designing processes , however they are restricted to only the software; consequently, to help the cycles of 
security prerequisites, we need a security requirements structure for CPSs.We humans live now in the era of digitization where software, system hardware, and sensors are working together as a close-knit unit over an established network. The main issue at hand is resolving to find the best method is obtaining an functional and successful security system for better Cyber-Physical Systems. There are a variety of reasons on the desire for an ideal security framework system is due to following threats of Eavesdropping, Compromised-key attack, Man-in-the-Middle Attack, Denial-of-Service Attack, Unauthorized Access, Protocol Failures, Physical Attack & Natural Disaster, and Radio Frequency Jamming\cite{Rehman}. The solutions in place to ensure the effective security requirements were the security goals such as the authentication, availability, integrity, and confidentiality. Security goals main objective were to focus on shielding the framework from the dangers and weaknesses and lessen risk factors.So far, the results portrayed were from an experiment to examine the CPS framework,where a smart car parking system was utilized in order to obtain promising results. With the assistance of this experiment they were able to identify 40 security requirements and these major security requirements were able to assist in the development of a secure smart car parking system

\end{abstract}



\section{Critique}
The article," An Effective Security Requirements Engineering
Framework for Cyber-Physical Systems" emphasizes on the importance of security goals to ensure a proper Cyber-Physical System, which is important to through both physical and social aspects of the framework. Regardless, what also should have been mentioned was how this can impact the framework from a global standpoint because not all countries utilize the same security framework as the US. Considering  how there are unique CPS systems set up throughout the layout of other countries which examines coverage areas, capacity, and security. What was particularly intriguing was the utilization of the smart car experiment to demonstrate the set of security goals that were developed from simply following the framework process guide, Identify security goals, Identify assets, Identify threats, Identify secure network communication, Identify endpoints hardware, Identify sensor communication medium, Perform Risk
assessment, and Elicit security requirements\cite{Rehman}.


\section{Synthesis}

There are a multitude of ways the experiment can be further developed to advance the framework applied to the CPS. For one instance developing a multi-authentication system can be proven helpful. Seeing how multiple steps are placed in a system for verifying the authenticity of a actual user to prevent Cyber-criminals from easily gain access. Additionally, another method would be to use a Raspberry Pi to host a web server as a local testing environment. From there the a multitude of security goals can be developed in order to obstruct threats and vulnerabilities and reduce risk factors \cite{Rehman} 

\bibliographystyle{ieeetr}
\bibliography{reference}

\end{document}
